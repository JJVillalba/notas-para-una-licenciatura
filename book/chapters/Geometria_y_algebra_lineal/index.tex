\chapter{Geometría y álgebra lineal} % (fold)
\label{cha:geometria_y_algebra_lineal}

\noindent \newthought{Ya para el siglo 17 los métodos sintéticos de la
geometría} no eran suficientes para abordar los nuevos problemas, fue Rene
Descartes el primero en proponer una forma de abordarlos realizando ``cálculos
geométricos'' con su novedoso sistema de coordenadas, al introducir esta nueva
herramienta podemos replantear muchos de resultados de la geometría euclidiana 
simplificando algunos y complicando otros, mostrándonos como cada problema tiene
requiere su  propio lenguaje.

\section{El plano cartesiano y el espacio euclídeo} % (fold)
\label{sec:el_plano_cartesiano_y_el_espacio_euclideo}

\section{Transformaciones lineales} % (fold)
\label{sec:transformaciones_lineales}

\section{Álgebra matricial} % (fold)
\label{sec:algebra_matricial}

% section algebra_matricial (end)


% section tranformaciones_lineales (end)


% chapter geometría_y_álgebra_lineal (end)