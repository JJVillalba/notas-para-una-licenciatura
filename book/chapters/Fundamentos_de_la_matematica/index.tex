\chapter{Fundamentos de las matemáticas} % (fold)
\label{cha:fundamentos_de_las_matematicas}

\noindent \newthought{Las matemáticas son complicadas, disculpa que sea} con lo
que empiece pero es verdad. Seguramente mas de una vez te hayas cruzado con
alguien que te diga que no las entiende, o es mas puede que tu mismo les tengas
``miedo'' por no decir algo mas, no voy a invalidar estas opiniones
achacándoselas al sistema educativo, o las herencias culturales, porque si bien
creo que ayudan al sentimiento de temor, las matemáticas son complicadas y por
eso vale la pena estudiarlas.

Las matemáticas son complicadas de una forma diferente a la filosofía, la
física, la economía, o cualquier otra disciplina que se te ocurra, en muchos
sentidos la matemática es única en sus complicaciones. Para empezar mucha de la
dificultad de las matemáticas no radica en su complejidad sino en su simpleza,
si tu le preguntas a tres expertos en economía política: ¿Cuál es el mejor
sistema económico? seguramente obtendrás tres respuestas distintas, todas
muy bien argumentadas y ninguna aparentemente equivocada, en cambio le preguntas
a cualquiera cuanto es \(2 + 2\) y puedes esperar \(4\) como respuesta con la
seguridad de que cualquier otra está equivocada. Si bien este ejemplo puede
parecer tendencioso, arreglado para dejar mal a las ciencias sociales, demuestra
una de las principales características de la matemática la \textbf{falsabilidad}

% chapter fundamentos_de_las_matematicas (end)