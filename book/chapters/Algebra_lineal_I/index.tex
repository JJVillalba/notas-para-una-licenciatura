\chapter{Álgebra lineal I} % (fold)
\label{cha:algebra_lineal_i}

\section{Espacios vectoriales} % (fold)
\label{sec:espacios_vectoriales}

\subsection{Espacios vectoriales y subespacios} % (fold)
\label{sub:espacios_vectoriales_y_subespacios}

\begin{definition}[Espacio vectorial]
	\label{def:espacio_vectorial}
	\HTMLclassTitle{Espacio vectorial}
	Sea \(\F\) un conjunto de escalares (usualmente \(\R\) o \(\C\)) y sea 
	\(\cV\) un conjunto con dos operaciones llamadas \textbf{adición} y 
	\textbf{adición	escalar}. Representamos la adición de \(\vec{v}, 
	\vec{w} \in \cV\) como \(\vec{v} + \vec{w}\), y la multiplicación de \(c
	\in \F\) y \(\vec{v}\) como \(c\vec{v}\).

	Si las siguientes condiciones se cumplen para todos \(\vec{v}, \vec{w}, 
	\vec{z} \in \cV\) y para todos \(c,d \in \F\), entonces decimos que \(\cV\) 
	es un \textbf{espacio vectorial} y llamamos a sus elementos 
	\textbf{vectores}:

	\begin{enumerate}
		\item \(\vec{a} + \vec{b} \in V\).

		\item \(\vec{v} + \vec{w} = \vec{w} + \vec{v}\).

		\item \((\vec{v} + \vec{w} ) +  \vec{z} = \vec{v} + (\vec{w} + 
		\vec{z}\)).

		\item Existe un elemento \(\vec{0} \in \cV\) tal que \(\vec{v} + 
		\vec{0} = \vec{v}\).

		\item Para todo vector \(\vec{v} \in \cV\) existe \(-\vec{v} \in \cV\)
		tal que \(\vec{v} + (-\vec{v}) = 0\).

		\item \(c \vec{v} \in \cV\).

		\item \(c(\vec{v} + \vec{w}) = c \vec{v} + c \vec{w} \).

		\item \((c+d) \vec{v} = c \vec{v} + d \vec{w}\).

		\item \(c(d \vec{v}) = (cd) \vec{v}\).

		\item \(1 \vec{v} = \vec{v}\).
	\end{enumerate}
	
\end{definition}

\begin{theorem}
	Suponga que \(\cV\) es un espacio vectorial y \(\vec{v} \in \cV\), entonces:
	\[ 0 \vec{v} = \vec{0} \quad \text{y} \quad	(-1) \vec{v} = -\vec{v} .\]
\end{theorem}

\begin{proof}
	Para mostrar que \(0 \vec{v} = \vec{0}\), usaremos con cuidado varias
	propiedades de la definición de espacio vectorial 
	\ref{def:espacio_vectorial}:
	\[ \begin{aligned}
		0 \vec{v} &= 0 \vec{v} + \vec{0} \\
			&= 0 \vec{v} + (0 \vec{v} + (- 0 \vec{v})) \\
			&= (0 \vec{v} + 0 \vec{v}) + (-0 \vec{v})\\
			&= (0 + 0) \vec{v} + (-0\vec{v}) \\
			&= 0\vec{v} + (-0\vec{v}) \\
			&= \vec{0}.
	\end{aligned} \]

	Ahora que sabemos que \(0 \vec{0}\), probar que \((-1)\vec{v} = -\vec{v}\)
	es mas sencillo:

	\[ \begin{aligned}
		\vec{0} &= 0\vec{v} \\
			&= (1 - 1) \vec{v} \\
			&= 1\vec{v} + (-1) \vec{v} \\
			&= \vec{v} + (-1)\vec{v} \\
	\end{aligned} .\]
	Se sigue que \((-1\vec{v}) = -\vec{v}\).
 \end{proof}

\subsubsection{Subespacios vectoriales} % (fold)
\label{ssub:subespacios vectoriales}

\subsubsection{Espacios generados, combinaciones lineales e independencia
 lineal} % % (fold)
\label{ssub:espacios_generados_combinaciones_lineales_e_independencia_lineal}

\subsubsection{Bases} % (fold)
\label{ssub:bases}

\subsection{Coordenadas y transformaciones lineales} % (fold)
\label{sub:coordenadas_y_transformaciones_lineales}

\subsubsection{Dimensiones y vectores de coordenadas} % (fold)
\label{ssub:dimensiones_y_vectores_de_coordenadas}

\subsubsection{Cambio de bases} % (fold)
\label{ssub:cambio_de_bases}

\subsubsection{Transformaciones lineales} % (fold)
\label{ssub:transformaciones_lineales}

\subsubsection{Propiedades de las transformaciones lineales} % (fold)
\label{ssub:propiedades_de_las_transformaciones_lineales}

\subsection{Isomorfismos y formas lineales} % (fold)
\label{sub:isomorfismos_y_formas_lineales}

\subsubsection{Isomorfismos} % (fold)
\label{ssub:isomorfismos}

\subsubsection{Formas lineales} % (fold)
\label{ssub:formas_lineales}

\subsubsection{Bilinealidad y otros} % (fold)
\label{ssub:bilinealidad_y_otros}

\subsubsection{Producto interno} % (fold)
\label{ssub:producto_interno}

\subsection{Ortogonalidad y adjuntas} % (fold)
\label{sub:ortogonalidad_y_adjuntas}

\subsubsection{Bases ortonormales} % (fold)
\label{ssub:bases_ortonormales}

\subsubsection{Transformaciones adjuntas} % (fold)
\label{ssub:transformaciones_adjuntas}

\subsubsection{Matrices unitarias} % (fold)
\label{ssub:matrices_unitarias}

\subsubsection{Proyecciones} % (fold)
\label{ssub:proyecciones}

\subsection{Mas acerca de la traza} % (fold)
\label{sub:mas_acerca_de_la_traza}

\subsubsection{Caracterización algebraica de la traza} % (fold)
\label{ssub:caracterizacion_algebraica_de_la_traza}

\subsubsection{Interpretación geométrica de la traza} % (fold)
\label{ssub:interpretacion_geometrica_de_la_traza}

\subsection{Suma directa y complemento ortogonal} % (fold)
\label{sub:suma_directa_y_complemento_ortogonal}

\subsubsection{La suma interna directa interna} % (fold)
\label{ssub:la_suma_interna_directa_interna}

\subsubsection{Complemento ortogonal} % (fold)
\label{ssub:complemento_ortogonal}

\subsubsection{La suma directa externa} % (fold)
\label{ssub:la_suma_directa_externa}

\subsection{La descomposición QR} % (fold)
\label{sub:la_descomposicion_qr}

\subsubsection{Afirmación y ejemplos} % (fold)
\label{ssub:afirmacion_y_ejemplos}

\subsubsection{Consecuencias y aplicaciones} % (fold)
\label{ssub:consecuencias_y_aplicaciones}

\subsection{Normas e isometrias} % (fold)
\label{sub:normas_e_isometrias}

\subsubsection{Las \(p\)-normas} % (fold)
\label{ssub:las_p_normas}

\subsubsection{De las normas devuelta a los productos internos} % (fold)
\label{ssub:de_las_normas_devuelta_a_los_productos_internos}

\subsubsection{Isometrias} % (fold)
\label{ssub:isometrias}

\section{Descomposición de matrices} % (fold)
\label{sec:descomposicion_de_matrices}

\subsection{La descomposiciones espectrales y de Schur} % (fold)
\label{sub:la_descomposiciones_espectrales y de Schur}

\subsubsection{Triangularización de Schur} % (fold)
\label{ssub:triangularizacion_de_schur}

\subsubsection{Matrices normales y la de descomposición espectral compleja} % 
\label{ssub:matrices_normales_y_la_de_descomposicion_espectral_compleja}

\subsubsection{La descomposición espectral real} % (fold)
\label{ssub:la_descomposicion_espectral_real}

\subsection{Matrices positivas} % (fold)
\label{sub:Matrices_positivas}

\subsubsection{Caracterización de las matrices positivas semidefinidas} % (fold)
\label{ssub:caracterizacion_de_las_matrices_positivas_semidefinidas}

\subsubsection{Dominancia diagonal y discos de Gershgorin} % (fold)
\label{ssub:dominancia_diagonal_y_discos_de_gershgorin}

\subsubsection{Libertad unitaria y descomposición PSD} % (fold)
\label{ssub:libertad_unitaria_y_descomposicion_psd}

\subsection{Descomposición en valores singulares} % (fold)
\label{sub:descomposicion_en_valores_singulares}

\subsubsection{Interpretación geométrica y los subespacios fundamentales} % 
\label{ssub:interpretacion_geometrica_y_los_subespacios_fundamentales}

\subsubsection{Relación con otras descomposiciones matriciales} % (fold)
\label{ssub:relacion_con_otras_descomposiciones_matriciales}

\subsubsection{El operador norma} % (fold)
\label{ssub:el_operador_norma}

\subsection{La descomposición de Jordan} % (fold)
\label{sub:la_descomposicion_de_jordan}

\subsubsection{Unicidad y similaridad} % (fold)
\label{ssub:unicidad_y_similaridad}

\subsubsection{Existencia y computo} % (fold)
\label{ssub:existencia_y_computo}

\subsubsection{Funciones matriciales} % (fold)
\label{ssub:funciones_matriciales}

\subsubsection{Funciones matriciales} % (fold)
\label{ssub:funciones_matriciales}

\subsection{Formas cuadráticas y secciones cónicas} % (fold)
\label{sub:formas_cuadraticas_y_secciones_conicas}

\subsubsection{Definitividad, ellipsoides y paraboloides} % (fold)
\label{ssub:definitividad_ellipsoides_y_paraboloides}

\subsubsection{Indefinitividad e hiperboloides} % (fold)
\label{ssub:indefinitividad_e_hiperboloides}

\subsection{Complemento de Schur y Cholesky} % (fold)
\label{sub:complemento_de_schur_y_cholesky}

\subsubsection{Complemento de Schur} % (fold)
\label{ssub:complemento_de_schur}

\subsubsection{La descomposición de Cholesky} % (fold)
\label{ssub:la_descomposicion_de_cholesky}

\subsubsection{Aplicaciones de la SVD} % (fold)
\label{ssub:aplicaciones_de_la_svd}

\subsubsection{Las pseudoinversa y los mínimos cuadrados} % (fold)
\label{ssub:las_pseudoinversa_y_los_minimos_cuadrados}

\subsubsection{Aproximaciones de bajo rango} % (fold)
\label{ssub:aproximaciones_de_bajo_rango}

\subsection{Continuidad y análisis matricial} % (fold)
\label{sub:continuidad_y_analisis_matricial}

\subsubsection{Conjuntos densos de matrices} % (fold)
\label{ssub:conjuntos_densos_de_matrices}

\subsubsection{Continuidad y funciones matriciales} % (fold)
\label{ssub:continuidad_y_funciones_matriciales}

\subsubsection{Trabajando con matrices no invertibles} % (fold)
\label{ssub:trabajando_con_matrices_no_invertibles}

\subsubsection{Trabajando con matrices no diagonalizables} % (fold)
\label{ssub:trabajando_con_matrices_no_diagonalizables}

\section{Tensores y multilinealidad} % (fold)
\label{sec:tensores_y_multilinealidad}

\subsection{El producto de Kronecker} % (fold)
\label{sub:el_producto_de_kronecker}

\subsubsection{Definición y propiedades básicas} % (fold)
\label{ssub:definicion_y_propiedades_basicas}

\subsubsection{Vectorización y la matriz de cambio} % (fold)
\label{ssub:vectorizacion_y_la_matriz_de_cambio}

\subsubsection{Los subespacios simétricos y antisimetricos} % (fold)
\label{ssub:los_subespacios_simetricos_y_antisimetricos}

\subsection{Transformaciones multilineales} % (fold)
\label{sub:transformaciones_multilineales}

\subsubsection{Definición y ejemplos básicos} % (fold)
\label{ssub:definicion_y_ejemplos_basicos}

\subsubsection{Listas} % (fold)
\label{ssub:listas}

\subsubsection{Propiedades de las transformaciones multilineales} % (fold)
\label{ssub:propiedades_de_las_transformaciones_multilineales}

\subsection{El producto tensorial} % (fold)
\label{sub:el_producto_tensorial}

\subsubsection{Motivación y definiciones} % (fold)
\label{ssub:motivacion_y_definiciones}

\subsubsection{Existencia y unicidad} % (fold)
\label{ssub:existencia_y_unicidad}

\subsubsection{Rango tensorial} % (fold)
\label{ssub:rango_tensorial}

\subsection{Funciones lineales de valores matriciales} % (fold)
\label{sub:funciones_lineales_de_valores_matriciales}

\subsubsection{Representación} % (fold)
\label{ssub:representacion}

\subsubsection{El producto Kronecker de funciones de valores matriciales} % 
\label{ssub:el_producto_kronecker_de_funciones_de_valores_matriciales}

\subsubsection{Funciones positivas y completamente positivas} % (fold)
\label{ssub:funciones_positivas_y_completamente_positivas}

\subsection{Polinomios homogéneos} % (fold)
\label{sub:polinomios_homogeneos}

\subsubsection{Potencias de formas lineales} % (fold)
\label{ssub:potencias_de_formas_lineales}

\subsubsection{Positividad semidefinida y suma de cuadrados} % (fold)
\label{ssub:positividad_semidefinida_y_suma_de_cuadrados}

\subsubsection{Formas bicuadraticas} % (fold)
\label{ssub:formas_bicuadraticas}

\subsection{Programación semidefinida} % (fold)
\label{sub:programacion_semidefinida}

\subsubsection{La forma de una programa semidefinido} % (fold)
\label{ssub:la_forma_de_una_programa_semidefinido}

\subsubsection{Interpretación geométrica y solución} % (fold)
\label{ssub:interpretacion_geometrica_y_solucion}

\subsubsection{Dualidad} % (fold)
\label{ssub:dualidad}