\chapter{Álgebra lineal I} % (fold)
\label{cha:algebra_lineal_i}

\section{Espacios vectoriales} % (fold)
\label{sec:espacios_vectoriales}

\begin{definition}[Espacio vectorial]
	\label{def:espacio_vectorial}
	\HTMLclassTitle{Espacio vectorial}
	Sea \(\F\) un conjunto de escalares (usualmente \(\R\) o \(\C\)) y sea 
	\(\cV\) un conjunto con dos operaciones llamadas \textbf{adición} y 
	\textbf{adición	escalar}. Representamos la adición de \(\vec{v}, 
	\vec{w} \in \cV\) como \(\vec{v} + \vec{w}\), y la multiplicación de \(c
	\in \F\) y \(\vec{v}\) como \(c\vec{v}\).

	Si las siguientes condiciones se cumplen para todos \(\vec{v}, \vec{w}, 
	\vec{z} \in \cV\) y para todos \(c,d \in \F\), entonces decimos que \(\cV\) 
	es un \textbf{espacio vectorial} y llamamos a sus elementos 
	\textbf{vectores}:

	\begin{enumerate}
		\item \(\vec{a} + \vec{b} \in V\).

		\item \(\vec{v} + \vec{w} = \vec{w} + \vec{v}\).

		\item \((\vec{v} + \vec{w} ) +  \vec{z} = \vec{v} + (\vec{w} + 
		\vec{z}\)).

		\item Existe un elemento \(\vec{0} \in \cV\) tal que \(\vec{v} + 
		\vec{0} = \vec{v}\).

		\item Para todo vector \(\vec{v} \in \cV\) existe \(-\vec{v} \in \cV\)
		tal que \(\vec{v} + (-\vec{v}) = 0\).

		\item \(c \vec{v} \in \cV\).

		\item \(c(\vec{v} + \vec{w}) = c \vec{v} + c \vec{w} \).

		\item \((c+d) \vec{v} = c \vec{v} + d \vec{w}\).

		\item \(c(d \vec{v}) = (cd) \vec{v}\).

		\item \(1 \vec{v} = \vec{v}\).
	\end{enumerate}
	
\end{definition}

\begin{theorem}
	Suponga que \(\cV\) es un espacio vectorial y \(\vec{v} \in \cV\), entonces:
	\[ 0 \vec{v} = \vec{0} \quad \text{y} \quad	(-1) \vec{v} = -\vec{v} .\]
\end{theorem}

\begin{proof}
	Para mostrar que \(0 \vec{v} = \vec{0}\), usaremos con cuidado varias
	propiedades de la definición de espacio vectorial 
	\ref{def:espacio_vectorial}:
	\[
	\begin{aligned}
		0 \vec{v} &= 0 \vec{v} + \vec{0} \\
			&= 0 \vec{v} + (0 \vec{v} + (- 0 \vec{v})) \\
			&= (0 \vec{v} + 0 \vec{v}) + (-0 \vec{v})\\
			&= (0 + 0) \vec{v} + (-0\vec{v}) \\
			&= 0\vec{v} + (-0\vec{v}) \\
			&= \vec{0}.
	\end{aligned}
	\]

	Ahora que sabemos que \(0 \vec{0}\), probar que \((-1)\vec{v} = -\vec{v}\)
	es mas sencillo:
\end{proof}