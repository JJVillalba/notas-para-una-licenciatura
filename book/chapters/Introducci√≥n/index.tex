\noindent Esta introducción, como casi todas, ha pasado por varios re-escrituras
si llevo bien la cuenta, esta debería ser la cuarta, un cambio de enfoque, 
un cambio de trabajo y un cambio de universidad han sido los motivos
principales, y creo que cada uno a dejado su impronta en el tono de lo que aquí
esta escrito, aportando un poco cada una a lo que son y representan estas notas.

Empece a escribir esto mientras estudiaba matemáticas en la USB, al principio
por el entusiasmo de haber aprendido \LaTeX y la triste realidad de que mi letra
deja mucho que desear, tomaba un libro y transcribía teoremas, pruebas y
definiciones y en proceso me los iba memorizando, además de quedarme unas
bonitas notas que luego revisar y compartir. Después de hacer esto con un par de
libros me di cuenta de que tenía ya bastante material, y empecé a agregar
pequeñas notas entre los teoremas y definiciones, notas históricas, pequeñas
explicaciones, y poquito a poquito lo que empezó como un seco conjunto de
notas, se volvió algo mucho mas interesante, esta fue la primera reescritura. Ya
un tiempo después, medio peleado con la universidad tome un trabajo como
programador y aprendí las arcanas artes de la programación web me convencí de
que tenía que hacer mi libro accesible desde la web, que podía de agregar
contenido interactivo y mejorar la presentación, que el contenido de lo merecía
y podía, esto supuso una segunda reescritura. Finalmente y en la etapa en la que
me encuentro al escribir esta cuarta introducción decidí cambiar de universidad,
y en el ínterin también se me daño mi computadora, así que una nueva reescritura
se hizo necesaria, una que como nuevo elemento introduce que ya por fin me
decidí a publicar esto.

Así hay algunas cosas importantes que señalar, este trabajo es por sobretodo
\emph{amateur} hecha por amor al contenido, con seriedad si, pero con poco
profesionalismo, nadie a verificado este trabajo, y por tanto advierto a todos
mis incautos lectores que lo que lean a pesar de mis mejores esfuerzos puede
estar plagado de mentiras. Con esto no espero asustar a nadie, de verdad no creo
que las situación sea tan mala, pero para leer este texto hace falta tener
siempre en mente que fue escrito por un \emph{estudiante mientras estudiaba} y
eso tiene sus riesgos. También invito a quien descubra algún error, impresión
o tenga alguna sugerencia a hacérmelo saber, el código fuente del libro esta
publicado en github (), donde puedes abrir un ``Issue''  con las observaciones
que creas pertinentes, con gusto las revisare.

Es posible que te hayas dado cuenta de que el indice del libro esta incompleto,
e incluso te pudo haber hecho gracia lo ambicioso del proyecto, a mí me lo hace
cada cierto tiempo, pero es que este libro es, y siempre sera un proyecto
incompleto, hace un rato acepte su naturaleza, y ahora ya solo sigo con ella,
tal vez en el mismo  espíritu que Donald Knut con \TeX, y es que este libro es
un diario intelectual cuyos horizontes se van expandiendo en la misma medida,
que voy descubriendo nuevas temas. Y como diario también es algo muy personal,
así que por favor no te asustes si consigues algo que no tiene nada ver con el
tema, algún poema, canción o vídeo puede que se te atraviese de cuando en vez
sin mucha relación con el tema, aunque intento que esto se mantenga en las
notas a al margen del libro.

