\chapter{Calculo integral en una variable I} % (fold)
\label{cha:calculo_integral_en_una_variable_i}

\noindent \newthought{Ya hemos resuelto la primera mitad del problema} y hemos
sentado las bases para abordar la segunda ¡Finalmente podremos hallar el área
debajo de una curva, dada su expresión! Parece poco pero este hecho tal ves
trivial nos dará acceso a otras muchas herramientas con las que atacar otros
problemas. 

% chapter calculo-integral-en-una-variable-i (end)