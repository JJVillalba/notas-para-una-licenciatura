% Packages

\usepackage{hyperref}

% Html conversion

% Dummy command to insert titles.
\newcommand{\HTMLclassTitle}[1]{}

% Document geometry
\usepackage{fancyhdr}
\fancyhead[LO]{Overleaf}
\fancyhead[RE]{Guides and tutorials}

\newlength{\hscale}
\newlength{\vscale}

%\makeatletter
\setlength{\hscale}{1mm}
\setlength{\vscale}{1mm}
%\makeatother


\usepackage{layout}
\usepackage{geometry}
 %\geometry{
	% asymmetric, % comment if the document will be two sides
 	%paperwidth=210\hscale,
	%paperheight=297\vscale,
 %}


% Layout #1: large margins
\newcommand{\marginlayout}{%
	\newgeometry{
		asymmetric,
	 	paperwidth=210\hscale,
		paperheight=297\vscale,
		top=27.4\vscale,
		bottom=27.4\vscale,
		inner=24.8\hscale,
		textwidth=107\hscale,
		marginparsep=6.2\hscale,
		marginparwidth=47.7\hscale,
	}%
	%\recalchead%
	%\widelayoutfalse%
}

\marginlayout

% Layout #2: small, symmetric margins
\newcommand{\widelayout}{%
	\newgeometry{
		top=27.4\vscale,
		bottom=27.4\vscale,
		inner=24.8\hscale,
		outer=24.8\hscale,
		marginparsep=0mm,
		marginparwidth=0mm,
	}%
	\recalchead%
	\widelayouttrue%
}

% Layout #3: no margins and no space above or below the body
\newcommand{\fullwidthpage}{%
	\newgeometry{
		top=0mm,
		bottom=0mm,
		inner=0mm,
		outer=0mm,
		marginparwidth=0mm,
		marginparsep=0mm,
	}%
	\recalchead%
	\widelayouttrue%
}


% Typography
\usepackage[osf,sc]{mathpazo}
\usepackage[scaled=0.9]{helvet}
\usepackage[scaled=0.85]{beramono} 
\usepackage[T1]{fontenc} %  8-bit encoding
\usepackage{textcomp} % supports the Text Companion fonts
\usepackage{microtype}
\usepackage{sidenotes}

\newcommand{\newthought}[1]{\textsc{#1}}

% Mathematics
\usepackage{amsmath}
\usepackage{amsfonts}
\usepackage{amsthm}

% Math Environments

\newtheorem{theorem}{Teorema}
\newtheorem{corollary}{Corolario}
\newtheorem{lema}{Lema}
\newtheorem{proposition}{Proposición}
\newtheorem{definition}{Definición}


\newcommand{\C}{\mathbb{C}} % Complex
\newcommand{\R}{\mathbb{R}} % Real
\newcommand{\Q}{\mathbb{Q}} % Rational
\newcommand{\F}{\mathbb{F}} % Arbitrary Field

\newcommand{\cV}{\mathcal{V}} % Vector Space
